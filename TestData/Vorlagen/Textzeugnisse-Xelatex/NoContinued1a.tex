%%%%%%%% Testing layout for text reports – 12.11.2018 %%%%%%%%

%%%% This allows page breaks within a report entry.
%%%% It has no ''Continued ...''.
%%%% It has drawn lines between individual entries.

%%%% Compile TWICE with xelatex to get correct page total.

\documentclass[12pt]{article}
\usepackage[a4paper,
top=0.5cm, bottom=1.5cm, left=2cm, right=2cm,
includeheadfoot, headsep=12mm, footskip=5mm] {geometry}
%% headsep is the gap before the first text line
\addtolength{\headheight}{1.2\baselineskip}

\usepackage{fontspec}
\setmainfont{Latin Modern Roman}
%\setsansfont{Arial}
%\setmonofont{Arial}
\usepackage{microtype}

%% DEBUGGING
%\errorcontextlines 10000
%\usepackage{showframe}
%%

\usepackage{xunicode}
\usepackage{polyglossia}
\setdefaultlanguage{german}
\usepackage{csquotes}
\MakeOuterQuote{"}

%%%% Header and Footer
\usepackage{lastpage}
\usepackage[explicit]{titlesec}
\usepackage{titleps}

\newpagestyle{main}[\small]{
%    \setheadrule{.5mm}%
    \renewcommand{\makeheadrule}{%
    \rule[-1.5\baselineskip]{\linewidth}{.5mm}}
    \sethead%
            {\begin{tabular}[t]{l}
                Zeugnis für\\
                \hspace{10mm}\textbf{Christian Eric Carstensen}
              \end{tabular}}%                               left
            {Klasse: \textbf{10}}%                          center
            {Schuljahr: \textbf{2015-2016}\\
              \hfill}%                                      right
    \setfootrule{.2mm}%
    \setfoot%
        {}%
        {Seite \thepage\ von \pageref{LastPage}}%
        {}%
}
\pagestyle{main}
%%% end Header and Footer

%%%% Draft / Final - switch
%% For draft version compile with xelatex [...] "\def\isdraft{1} \input{<filename>}"
%% <filename> needs no ".tex" extension.
%% The following is just an example for using \isdraft, it is not used.
\ifdefined\isdraft
  \newcommand{\InDraft}{DRAFT}
\else
  \newcommand{\InDraft}{}
\fi

%% Tolerate wider spaces to avoid over-long lines
\tolerance=1000
\emergencystretch 1em
%\emergencystretch 3em

%% Stuff needed for dealing with layout
\usepackage{ifthen}
\usepackage{refcount}
\newcounter{oldpage}	% used for page number of last block
\newcounter{mycntl}		% for numbering the blocks

\setlength{\parindent}{5mm}

%% Stuff for tweaking page breaks
%\usepackage{needspace}
%\widowpenalties 3 10000 10000 150
%\clubpenalties 3 10000 10000 150
%\widowpenalty=10000
\widowpenalties 2 10000 150

\makeatletter
\titleformat{\section}
  {\normalfont\bfseries\large}
  {}
  {0pt}
%  {\gdef\@section@title@{\thesection\quad#1 (continued)}#1}
 {\gdef\@section@title@{#1 (Fortsetzung)}%
 #1}
%\makeatother

%\makeatletter
%\let\@section@title@\relax% Sectional heading storage
%\def\print@section@title@{%
%  {\hfill
%  \par\quad\normalfont\bfseries\large\@section@title@}\par\vspace{2ex}%
%}
%\EveryShipout{%
%  \ifdim\pagetotal>\pagegoal% There is content overflow on this page
%     \ifthenelse{\equal{\thepage}{\getpagerefnumber{marka\themycntl}}}
%    {\ifdef{\blockname}
%      {\aftergroup\print@section@title@% Reprint/-insert sectional heading
%      }
%      {}
%    }
%    {}
%  \fi%
%}
\makeatother

\newif\iftopbar

\newcommand{\heading}[1]{%
\noindent\begin{minipage}[b][1.4cm][b]{\textwidth}
  \iftopbar
    \noindent\hfil\rule{0.7\textwidth}{0.5pt}\hfil
  \fi
  \vfill
%\textsc doesn't work (for this font?)
  \hspace{7mm}\large \textsc{\textbf{#1}}  --- \the\pagetotal : \the\pagegoal
  \vspace{1mm}
  \end{minipage}
}

\newsavebox{\boxname}
\newlength{\boxlength}

%%%%MACRO: A report entry block
\newcommand{\myblock}[3]{%
%\noindent
\pagespace
\ifdim\pagetotal=0pt
  % At top of page
  \topbarfalse
  ~\vskip -8mm
\else
  \ifdim\spaceleft<3cm
    \newpage
    ~\vskip -8mm
    \topbarfalse
  \else
    \topbartrue
  \fi
\fi
%\needspace{3cm}
\stepcounter{mycntl}
%
%% If the previous entry is on the same page, add a separator line:
%+\ifthenelse{\equal{\theoldpage}{\getpagerefnumber{marka\themycntl}}}
%{%%then
% Add a separator line.
% \raisebox allows the placement to be adjusted vertically without
% using up extra vertical space.
%+  {\centering
%+  \raisebox{9mm}[0pt][0pt]{\rule{14cm}{.3mm}}\par
%+  }
%+}
%+{%%else
  %% This clause must use the same vertical space as the <if> clause
%+  \hfill
%+}
%% Undo the vertical space used in the <ifthenelse> block above:
%+\vspace*{-\baselineskip}\vspace*{-\parskip}

%% Block title:
\newcommand{\blockname}{#1}
\label{marka\themycntl}
\heading{#1}
%
%text body
#2
%% The teacher's name should appear below, but never on the next
%% page, so a page break must be avoided here. This is one possibility:
%\\*
\label{markb\themycntl}
\savebox{\boxname}{#3}
\settowidth{\boxlength}{\usebox{\boxname}}
\addtolength{\boxlength}{3cm minus 2cm}
%\addtolength{\boxlength}{2cm}
\hfill\rule[-6mm]{1pt}{6mm}\makebox[\boxlength][r]{\raisebox{-5mm}[0pt][0pt]{\usebox{\boxname}}}	%teacher
%
%\convertto{mm}{\the\boxlength}

\setcounter{oldpage}{\getpagerefnumber{markb\themycntl}}
\let\blockname\undefined
%\vspace{16mm}
\vspace{5mm}
%%\vskip 16mm	% \vspace seems unreliable in suppressing the space at page breaks
}
%%%% End report entry block

%%%%MACRO: Command to show an intro block in italic
\newcommand{\Intro}[1]{\fontshape{it}\selectfont{}#1\fontshape{\shapedefault}\selectfont{}}

\newdimen\spaceleft
\def\pagespace{%
  \ifdim\pagetotal=0pt
      \spaceleft=\vsize
  \else
     \spaceleft=\pagegoal
     \advance\spaceleft by -\pagetotal
  \fi
%  \the\pagetotal : \the\spaceleft
}

\makeatletter
\def\convertto#1#2{\strip@pt\dimexpr #2*65536/\number\dimexpr 1#1}
\makeatother

%%%%%%%%%%%%%%%%%%%%%%%%%%%%%%%
\begin{document}
\myblock{Mathematik}{%
\Intro{%
Im Unterricht zur Mathematik wurden Grundlagen zur Algebra und zur Stereometrie
bearbeitet; das Jahr endete mit einer Epoche zur Trigonometrie. Es wurden
Verfahren zur Lösung von quadratischen Gleichungen erarbeitet und auch Verfahren
für die Lösung von linearen Gleichungssystemen. Hierbei wurde Wert darauf
gelegt, dass die bestehenden Lücken geschlossen wurden.\par
}
Im Unterricht zur Mathematik wurden Grundlagen zur Algebra und zur Stereometrie
bearbeitet; das Jahr endete mit einer Epoche zur Trigonometrie.Christian kam am
Anfang des Schuljahres in die Gruppe. Es zeigte sich, dass er die Verfahren
sicher beherrscht und er verstand die behandelten Themen schnell. Er konnte die
Rechentechniken erfolgreich einsetzen, anscheinend ohne besondere Anstrengung.
Hier und da tauchten jedoch Unsicherheiten auf, z.\,B. im Umgang mit
Bruchgleichungen. Seine Leistungen lagen insgesamt im guten Bereich.}%
{Fritz Folter}

\myblock{Biologie}{%
\Intro{%
In der diesjährigen Biologie-Epoche beschäftigten wir uns mit den Organsystemen
des menschlichen Körpers. Besonders intensiv erarbeiteten wir uns das
Immunsystem, das Herz-Kreislaufsystem und das Atmungssystem. Anstelle einer
abschließenden Klassenarbeit informierten sich die Schüler selbstständig über
ein sachbezogenes Thema und präsentierten ihre Ergebnisse in Form eines
Referats.\par
}
Christian beteiligte sich von sich aus nicht am Unterrichtsgespräch und
verfolgte den Unterricht so eher passiv. Er hielt kein Referat, da er an diesem
Tag nicht den Unterricht besuchte. Als Ersatzleistung schrieb Christian eine
Klassenarbeit. Anhand des Ergebnisses wurde sehr deutlich, dass er sich nicht
hinreichend mit den Themen der Epoche auseinandergesetzt hatte. In der Epoche
erarbeitete Zusammenhänge zur Funktionsweise der Organe konnten von Christian
nicht erklärt bzw. beschrieben werden. Auch bei der Führung seines Epochenheftes
muss sich Christian noch deutlich steigern. Seinem aktuellen Heft mangelt es an
einer brauchbaren Form, außerdem sind die Texte oft unvollständig. Dies betrifft
insbesondere die Eigenleistungen. Die anatomischen Zeichnungen wurden nicht mit
der nötigen Sorgfalt angelegt und sind daher in den entscheidenden Details oft
fehlerhaft. In Zukunft muss Christian sich mit mehr Aktivität und Energie am
Unterricht beteiligen und größere Anstrengungen unternehmen, den Stoff
gedanklich zu durchdringen.
}%
{Mark Essich}

\myblock{Chemie}{%
\Intro{%
In der Chemie-Epoche des 10. Schuljahres haben wir uns mit Säuren, Laugen und
Salzen beschäftigt. Dabei haben wir die Charakteristika verschiedener Säuren und
Laugen herausgearbeitet, sowie die Beziehung zwischen den Salzen und den Säuren
und Laugen ergründet.\par
}
Christian beteiligte sich leider mündlich überhaupt nicht am Unterricht und
versäumte es auch mehrfach seine Hausaufgaben zu erledigen. Dementsprechend war
auch sein Epochenheft nur unvollständig geführt und enthielt inhaltliche Fehler.
In den schriftlichen Zwischentests konnte Christian die grundlegenden
Zusammenhänge des Epochenstoffs reproduzieren, in der abschließenden
Klassenarbeit zeigte sich jedoch, dass bei Christian noch größere Wissenslücken
bestanden. In Zukunft muss sich Christian gedanklich aktiver mit den
Unterrichtsinhalten beschäftigen.
}%
{Mark Essich}

\myblock{Physik}{%
\Intro{%
In der Epoche zur Physik stand die Mechanik auf dem Programm. Es wurden Pendel
als Grundlage der Zeitmessung betrachtet und Federn zur Kraftmessung. Es wurden
die Statik und die Dynamik behandelt. Ebenso wurde die Überlagerung von
Bewegungen betrachtet.\par
}
Christian beteiligte sich recht angemessen am Unterricht. Sein Epochenheft ist
vollständig, aber nicht immer ganz ordentlich. An der schriftlichen Arbeit nahm
er nicht teil. Daher ist eine abschließende Beurteilung nicht möglich.
}%
{Dr. Birgit Sprengstoff}

\myblock{Deutsch}{%
\Intro{%
In der ersten Epoche des Jahres setzten wir uns mit dem Nibelungenlied und
dessen Rezeption auseinander und unternahmen eine kleine Studienreise nach
Weimar und zur Gedenkstätte Buchenwald. Der Roman „Sansibar oder Der letzte
Grund“ von Alfred Andersch sowie journalistische und lyrische Texte zum Thema
Flucht standen im Mittelpunkt der zweiten Epoche. Verschiedene Methoden der
Texterschließung wurden geübt und angewendet.\par
}
Christian entwickelte im Laufe des Schuljahres immer mehr Interesse für die
Themen und arbeitete mündlich zunehmend aktiver mit. Seine schriftlichen
Leistungen waren wechselhaft: Dass er einen Sinn für das Wesentliche hat, konnte
er in seinen Texten zur Nibelungenliedepoche unter Beweis stellen. In der
zweiten Epoche gab er ein unvollständiges, nur skizzenhaft angelegtes Heft ab.
Es wäre wichtig, wenn Christian im kommenden Schuljahr das Verfassen von
Aufsätzen ernster nähme, vor allem muss er üben, sich gründlicher mit den
Aufgabenstellungen auseinanderzusetzen. Christians Rechtschreibung ist in
Ordnung, grammatikalisch zeigen sich noch Unsicherheiten, stilistisch kann er
noch einiges dazulernen.
}%
{Gudrun Unterbach}

\myblock{Englisch}{%
\Intro{%
Zu Beginn des Schuljahres lasen wir die Lektüre zum gleichnamigen Film "Cry
Freedom". Es sind die Erinnerungen des südafrikanischen Journalisten Donald
Woods an seinen Freund Steven Biko. Biko setzte sich für die friedliche
Überwindung des rassistischen Apartheid Regimes ein und wurde dafür zu Tode
gefoltert. Wir beschäftigten uns auch mit der Geschichte Südafrikas, und
versuchten die Wurzeln der Rassentrennung zu verfolgen. Während wir den ersten
Teil der Lektüre gemeinsam lasen und bearbeiteten, wurde der zweite Teil
eigenständig von den Schülern und Schülerinnen erarbeitet. Jeweils zu zweit
wurde ein Kapitel vorgestellt. Die Mitschüler mussten anhand von Fragen, Rätseln
und Grammatikübungen ihre Aufmerksamkeit unter Beweis stellen. Hier hat wirklich
fast jede/r ihr/sein Bestes gegeben, und das selbstständige Arbeiten zeigte, was
die Schüler und Schülerinnen in der Fremdsprache bereits alleine leisten können.
Musikalisch untermalte dabei Peter Gabriels Song "Biko" unsere Arbeit, der fast
den Status eines Klassenschlagers erreichte.Zum Abschluss dieses Themas
beschäftigten wir uns noch mit Nelson Mandela, Ken Saro Wiwa und anderen
"Helden", die wiederum durch Referate vorgestellt wurden.\par
Im zweiten Halbjahr widmeten wir uns Großbritannien, wir lasen Texte über die
Monarchie und die Zeit als Kolonialreich. So ergab sich der Bezug zum Thema
"Multiculturalism in Britain", denn wir hatten gelernt, warum unter anderen
viele Inder und Pakistani in Großbritannien leben. In unserer Lektüre "Bend it
like Beckham" ging es dann um eine fußballverrückte und sehr talentierte junge
Inderin, die es schafft, sich über kulturelle Konventionen hinwegzusetzen. Wir
tauchten tief in die indische Kultur ein, lernten über "arranged marriages",
Religion und Küche dieses Kontinents. Neben Wiederholungen wichtiger
Grammatikaspekte konnten die Schüler in diesem Jahr neue wichtige Vokabeln und
Formulierungen erlernen und viele haben sich beim Formulieren englischer Texte
sehr verbessern können.\par
}
Christian, Sie haben sich die meiste Zeit bemüht, sowenig wie möglich zu tun.
Schade! Mit etwas mehr Einsatz hätten Sie sicherlich viel mehr erreichen können.
Auf Anfrage zeigten Sie nämlich, dass Sie sich sehr wohl am Unterrichtsgespräch
mit guten Gedanken beteiligen können. Auch die Erledigung der Hausarbeiten war
sehr unregelmäßig. Für das kommende Jahr ist mehr Einsatz gewünscht!
}%
{James Bond}

\myblock{Französisch}{%
\Intro{%
}
Christian hat sehr ruhig am Französischunterricht teilgenommen. Er meldete sich
so gut wie nie und konnte mündlich kaum Fortschritte erzielen.  Es mangelte
Christian an Fleiß und Kontinuität. In der Lektürearbeit fasste er zum Beispiel
kaum ein Kapitel selbstständig zusammen.  Vokabeln und grammatikalische
Neuinhalte übte er nicht ausreichend, sodass er nur knapp zufriedenstellende bis
mangelhafte Arbeiten schrieb.  Er fing auch zu spät an, zu lernen. Im nächsten
Schuljahr muss Christian sehr viel tun und eine andere Arbeitshaltung vorweisen.
}%
{Candice d‘Amour}

\myblock{Geschichte}{%
\Intro{%
In den beiden Geschichtsepochen wurde die Menschheitsgeschichte von den Jägern
und Sammlern über die frühen Hochkulturen bis zum antiken Griechenland
exemplarisch behandelt.\par
}
Christian hat sich am Unterrichtsgespräch kaum beteiligt. In der zweiten Arbeit
hat er sich merklich gesteigert, indem er in allen Bereichen vielversprechende
Ansätze gezeigt hat. Zum Schluss aber wurden die Ausführungen viel zu knapp. Die
Hefte enthalten wesentliche Beiträge, sind aber beide ziemlich lückenhaft und
hätten ordentlicher geführt werden müssen.
}%
{Dr. Birgit Sprengstoff}

\myblock{Religion}{%
\Intro{%
}
Christian hat still und aufmerksam am Religionsunterricht der
Christengemeinschaft teilgenommen.
}%
{Dr. Gisela Kerzengrad}

\myblock{Sozialkunde}{%
\Intro{%
Die Sozialkunde beschäftigte sich im 10. Schuljahr mit dem Verfassungsaufbau
unseres Staates, mit Parteien, Wahlen und der Demokratie in ihren
Entwicklungsmöglichkeiten, wobei z. B. Systeme für Volksentscheide oder ein
bedingungsloses Grundeinkommen diskutiert wurden.\par
}
Christian beteiligte sich dabei immer wieder auch aktiv und mit sinnvollen
Beiträgen am Unterrichtsgespräch. Die Parteienvorstellung zur FDP war gut
strukturiert, aber nicht sehr umfänglich vorbereitet, die Darstellung zum
Zukunftskonzept der Volksentscheide gelang dann recht gut. Insgesamt ist also
ein noch bewussterer Arbeitszugriff zu wünschen, dann wird Christian sein
merkbares Sachinteresse auch in gute Lernerfolge umsetzen können.
}%
{Arthur Benommen}

\myblock{Geographie}{%
\Intro{%
In der Geographie-Epoche lernten die Schüler Phänomene des Wetters und des
Klimas kennen. Sie verfolgten das tägliche Wettergeschehen in Zusammenhang mit
der Verteilung von Hoch- und Tiefdruckgebieten und erstellten dazu eine eigene
Graphik.\par
}
Christian arbeitete im Unterricht  aufmerksam, interessiert und mündlich
teilweise aktiv mit. Seine Beiträge waren ansprechend, den Wetterbericht hat er
angemessen zusammengestellt. Das Epochenheft hat er aber nicht abgegeben. In der
Abschlussarbeit erreichte er etwa die Hälfte der maximal möglichen Leistung und
zeigte damit insgesamt ein befriedigendes Epochenergebnis.
}%
{Arthur Benommen}

\myblock{Sport}{%
\Intro{%
}
Christian setzte sich mit der bewussten Beherrschung des Schwunges auseinander.
Anhand der Diskus\mbox{-,} Hockey\mbox{-,} Turn- und Basketballepoche konnte er
die Aspekte des Körperschwunges erleben. In der Diskuswurfepoche war er bemüht,
sich die Schwungtechnik anzueignen. Durch ein ausdauernderes Üben, hätte die
Technik an weiterführender Präzision gewinnen können, doch leider zeigte
Christian in dieser Hinsicht kein Engagement. In der Hockey- und
Basketballepoche beschäftigte sich Christian schon weitgehend aufmerksamer und
intensiver mit den einzelnen Spieltechniken. Folglich setzte er diese dynamisch
und  aktiv in den Spielen um. In der Turnepoche sollte Christian sich
eigenständig eine Übungsabfolge auf dem großen Trampolin und an den Ringen
erarbeiten. Hierfür trainierte er, teilweise arbeitsam, an den verschiedenen
Turnelementen. Um allerdings die Fähigkeiten für diese Geräte zu steigern,
müssen ausdauerndere und beständigere Trainingseinheiten erfolgen. Im nächsten
Schuljahr sollte Christian daher ein intensiveres Training in allen Bereichen
verfolgen, um somit an Präzision in seinen Körperbewegungen zu gewinnen und
verstärkte Tatkraft im Unterricht zeigen zu können.
}%
{Mareike Kletterer}

\myblock{Schnitzen}{%
\Intro{%
Die Schüler haben an einer Holzskulptur gearbeitet, die die Spiralform und ihre
unterschiedlichen Flächenbewegungen zum Thema hatte.\par
}
Christian Eric hat sich in seiner Arbeit besonders mit der Komposition von
Lochbildung und Durchbrüchen auf den senkrechten Flächen beschäftigt. Im Laufe
der Arbeit zeigten sich sanfte Flächenbewegungen, die in schöner Weise mit den
scharfen Hell-Dunkelkontrasten der schwarzen Löcher korrespondierten. Eine
interessante und räumlich von allen Seiten gut nachvollziehbare Arbeit.
}%
{Berndt Graziosi}

\myblock{Weben}{%
\Intro{%
}
Christian gelang die Koordination zwischen Hand und Fuß am Spinnrad nach langer
Übungszeit und Ausdauer. So stellte er dann zum Ende der Zeit noch einen fast
gleichmäßig verdrehten Faden her. Christian arbeitete ruhig, ausdauernd und nahm
Verbesserungsvorschläge an.\par
Christian webte auf einem Flachwebstuhl einen langen Baumwollschal und einen
kleinen Läufer sowie einen kleinen Wollteppich an einem Hochwebstuhl. Ihm fiel
die Arbeit mit den vielen Fäden nicht leicht. Kontinuierlich führte er alle
Anweisungen zu den Arbeitsschritten, die nötig sind um einen Webstuhl
einzurichten, korrekt aus. Die Mechanik des Webstuhls hat er verstanden. Seinen
Baumwollschal webte er mit einem gleichmäßigen Anschlag im Gewebe und sauberen
Webkanten ab. Für das zweite Gewebe war der Webstuhl mit der Kette schon
eingerichtet. Christian stellte auch dieses Gewebe fleißig und selbstständig
her.
}%
{Adelheid Altmode}

\myblock{Medienkunde}{%
\Intro{%
Grundlagen des Journalismus, der Filmanalyse und der Umgang mit den digitalen
Medien wurden in der Medienkunde-Epoche vermittelt, in der die Schülerinnen und
Schüler ein Thema selbst erschlossen und im Unterricht präsentierten und
außerdem eine eigene Reportage verfassten.\par
}
Christians Portfolio war gut gestaltet, sein Referat über Edward Snowden
inhaltlich interessant. Christian hätte aber die dafür verwendeten Quellen
nennen müssen -- der Text orientiert sich stark an Wikipedia-Einträgen.
Wahrscheinlich hätte Christian noch mehr von dem Thema gehabt, wenn er das Buch
über Snowden dazu ganz gelesen hätte.
}%
{Gudrun Unterbach}

\myblock{Betriebspraktikum}{%
\Intro{%
}
Christian absolvierte sein Betriebspraktikum in der Firma Klassic Kars in
Madendorf, die historische Fahrzeuge restauriert und wartet. Dieses
dokumentierte er in einer sehr ansprechend gestalteten Mappe. Ebenso waren in
dieser seine Tätigkeiten gut und aussagekräftig dokumentiert. Christian brachte
sich gut in das Werkstatt-Team ein und arbeitete an einigen Fahrzeugen bzw.
Teilen mit. Insgesamt kann das Praktikum als voller Erfolg gewertet werden.
}%
{Dr. Birgit Sprengstoff}


\end{document}